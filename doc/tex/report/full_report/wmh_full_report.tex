\documentclass[]{project_report}
\author{%
	Marcin Lembke\\
	\texttt{\href{mailto:M.Lembke@stud.elka.pw.edu.pl}%
			{\nolinkurl{M.Lembke@stud.elka.pw.edu.pl}}}
	\and
	Marcin Malesa\\
	\texttt{\href{mailto:M.Malesa@stud.elka.pw.edu.pl}%
			{\nolinkurl{M.Malesa@stud.elka.pw.edu.pl}}}
}
\supervisor{dr inż. Piotr Bilski}
\title{Algorytm przeszukiwania z tabu do rozwiązywania układanki Sudoku (PB5)}
\subtitle{Sprawozdanie końcowe}
\course{Współczesne metody heurystyczne}
\coursecode{WMH}
\university{Politechnika Warszawska}
\faculty{Wydział Elektroniki i Technik Informacyjnych}

\addbibresource{bibliography.bib}

\begin{document}
	\maketitle
	
	\section{Wstęp}
		W projekcie zaimplementowano algorytm rozwiązywania łamigłówek Sudoku z wykorzystaniem metaheurystyki tabu. Problem generowania plansz zdefiniowano jako problem spełniania ograniczeń (ang. \textit{constraint satisfaction problem}, CSP) a rozwiązanie (generowanie plansz) zaimplementowano w języku Python z wykorzystaniem biblioteki \texttt{python-constraint}.
		
		Rozwiązywanie plansz sudoku zaimplementowano w języku C++ z wykorzystaniem biblioteki standardowej.

	\section{Analiza eksperymentalna programu}
		\begin{figure}
			\centering
			\includegraphics[width=\textwidth]{iterations_fixed.pdf}
			\caption{Wykres przedstawiający średnią liczbę iteracji w zależności od liczby początkowo uzupełnionych pól w łamigłówce}
		\end{figure}
		
	\printbibliography
\end{document}
