\documentclass[]{project_report}
\usepackage{color}
\usepackage{listings}
\definecolor{mygreen}{rgb}{0,0.6,0}
\definecolor{mygray}{rgb}{0.5,0.5,0.5}
\definecolor{mymauve}{rgb}{0.58,0,0.82}
\lstset{ %
	backgroundcolor=\color{white},   % choose the background color; you must add \usepackage{color} or \usepackage{xcolor}; should come as last argument
	basicstyle=\footnotesize,        % the size of the fonts that are used for the code
	breakatwhitespace=false,         % sets if automatic breaks should only happen at whitespace
	breaklines=true,                 % sets automatic line breaking
	captionpos=b,                    % sets the caption-position to bottom
	commentstyle=\color{mygreen},    % comment style
	deletekeywords={...},            % if you want to delete keywords from the given language
	escapeinside={\%*}{*)},          % if you want to add LaTeX within your code
	extendedchars=true,              % lets you use non-ASCII characters; for 8-bits encodings only, does not work with UTF-8
	frame=none,	                   % adds a frame around the code
	keepspaces=true,                 % keeps spaces in text, useful for keeping indentation of code (possibly needs columns=flexible)
	keywordstyle=\color{blue},       % keyword style
	language=Octave,                 % the language of the code
	morekeywords={*,...},           % if you want to add more keywords to the set
	numbers=none,                    % where to put the line-numbers; possible values are (none, left, right)
	numbersep=5pt,                   % how far the line-numbers are from the code
	numberstyle=\tiny\color{mygray}, % the style that is used for the line-numbers
	rulecolor=\color{black},         % if not set, the frame-color may be changed on line-breaks within not-black text (e.g. comments (green here))
	showspaces=false,                % show spaces everywhere adding particular underscores; it overrides 'showstringspaces'
	showstringspaces=false,          % underline spaces within strings only
	showtabs=false,                  % show tabs within strings adding particular underscores
	stepnumber=2,                    % the step between two line-numbers. If it's 1, each line will be numbered
	stringstyle=\color{mymauve},     % string literal style
	tabsize=3,	                   % sets default tabsize to 2 spaces
	title=\lstname                   % show the filename of files included with \lstinputlisting; also try caption instead of title
}

\author{%
	Marcin Lembke\\
	\texttt{\href{mailto:M.Lembke@stud.elka.pw.edu.pl}%
		{\nolinkurl{M.Lembke@stud.elka.pw.edu.pl}}}
	\and
	Marcin Malesa\\
	\texttt{\href{mailto:M.Malesa@stud.elka.pw.edu.pl}%
		{\nolinkurl{M.Malesa@stud.elka.pw.edu.pl}}}
}
\supervisor{dr inż. Piotr Bilski}
\title{Algorytm przeszukiwania z tabu do rozwiązywania układanki Sudoku (PB5)}
\subtitle{Sprawozdanie cząstkowe}
\course{Współczesne metody heurystyczne}
\coursecode{WMH}
\university{Politechnika Warszawska}
\faculty{Wydział Elektroniki i Technik Informacyjnych}

\addbibresource{bibliography.bib}

\begin{document}
	\maketitle
	
	\section{Sudoku}
	Sudoku to gra logiczna, której celem jest uzupełnienie planszy, zazwyczaj o rozmiarze 9 na 9 pól, w taki sposób aby w każdym wierszu, kolumnie oraz wyznaczonym bloku 3 na 3 pola, znalazło się po jednej cyfrze z zakresu 1 do 9.
	
	\section{Generowanie plansz}
	Pierwszym problemem, który wyszczególniliśmy w sprawozdaniu wstępnym jest generowanie plansz Sudoku. Podaliśmy wtedy dwa pomysły na generowanie plansz: pierwszy polegajacy na generowaniu z wykorzystaniem bazy ,,dobrych'' plansz, oraz drugi, bardziej złożony zaprezentowany w artykule [tutaj odnośnik do artykułu].
	Ostatecznie zdecydowaliśmy się wzorować na metodzie zaprezentowanej w artykule, lecz z pewnymi zmianami:
	\begin{enumerate}
		\item Zdecydowaliśmy się przedstawić, na potrzeby generowania plansz, problem Sudoku jako problem spełniania ograniczeń (ang. \textit{constraint satisfaction problem}, CSP). To podejście wydaje się być dobrym modelem układanki Sudoku.
		\item Zrezygnowaliśmy z założenia o jednoznacznym rozwiązaniu planszy, ponieważ założenie to, mimo że jest czasami pożądane w przypadku układanki wygenerowanej do rozwiązania przez człowieka, nie jest wymagane do zrealizowania celu projektu, czyli zaprojektowania i implementacji algorytmu przeszukiwania z tabu, rozwiązującego układankę Sudoku. Innymi słowy interesuje nas sam fakt, znalezienia jakiegokolwiek rozwiązania przez zaimplementowany algorytm.
		\item Uprościliśmy metodę oceny trudności planszy do jednego czynnika: liczby nieuzupełnionych pól.
		\item Uprościliśmy metodę ,,robienia dziur'' w układance do zwykłego losowania wypełnionych pól i ich usuwania.
	\end{enumerate}
	\subsection{Sudoku jako CSP}
	Problem spełnienia ograniczeń (CSP) można przedstawić jako trójkę \((X , D, C)\):
	\begin{align*}
		X &= \left\lbrace X_{1}, \ldots, X_{n} \right\rbrace,\\
		D &= \left\lbrace D_{1}, \ldots, D_{n} \right\rbrace,\\
		C &= \left\lbrace C_{1}, \ldots, X_{m} \right\rbrace,
	\end{align*}
	gdzie: \(X\) jest zbiorem zmiennych, \(D\) jest zbiorem dziedzin tych zmiennych oraz \(C\) jest zbiorem ograniczeń.
	
	\begin{equation*}
		C \land x_{1} \in D(x_{1}) \land \ldots \land x_{n} \in D(x_{n})
	\end{equation*}
	Rozwiązanie układanki Sudoku o rozmiarze 9 na 9 można przedstawić jako CSP z następującymi ograniczeniami:
	\begin{enumerate}
		\item W każdym wierszu musi znaleźć się dokładnie po jednej cyfrze od 1 do 9.
		\item W każdej kolumnie musi znaleźć się dokładnie po jednej cyfrze od 1 do 9.
		\item W każdym kwadracie \(3 \times 3\) musi znaleźć się dokładnie po jednej cyfrze od 1 do 9.            
	\end{enumerate}
	Oczywiście w tym wypadku zmiennych jest 81 i dziedzina każdej zmiennej to \(\left\lbrace 1, 2, \ldots, 9 \right\rbrace\).
	
	Algorytm generowania układanek Sudoku wygląda następująco:
	\begin{enumerate}
		\item Stwórz instancję problemu CSP.
		\item Dodaj zmienne -- wygeneruj pustą planszę.
		\item Dodaj ograniczenia.
		\item Wykorzystaj \textit{solver} CSP do rozwiązania planszy.
		\item Wylosuj uzupełnione pola do usunięcia.
	\end{enumerate}
	
	Generowanie plansz zostało zaimplementowane w języku Python w wersji 3 z wykorzystaniem biblioteki \textit{python-constraint}\footnote{\url{https://github.com/python-constraint/python-constraint}}, dostarczającej m.in. \textit{solver} CSP.
	
	\section{Przeszukiwanie tabu}
	Po obmyśleniu sposobu na generowanie rozwiązywalnych plansz Sudoku, pora przystąpić do zaprojektowania algorytmu przeszukiwania z tabu. Pierwszym, aczkolwiek trywialnym zadaniem jest ustalenie sposobu reprezentacji plansz. Zdecydowaliśmy się robić to za pomocą zwykłej dwuwymiarowej tablicy.
	\begin{lstlisting}[language=C++]
	class Node	// Klasa reprezentujaca pojedyncze pole na planszy
	{
	unsigned int value;   // Wartosc pola
	bool startingNode;   // Czy pole jest polem poczatkowym?
	
	(...)
	};
	
	class Sudoku	// Klasa reprezentujaca pojedyncza plansze Sudoku
	{
	Node map[9][9];
	
	(...)
	};
	\end{lstlisting}
	Kolejnym ważnym zadaniem było zdefiniowanie relacji sąsiedztwa na parach elementów w przestrzeni rozwiązań, która będzię obejmować całą jej dziedzinę. Zdecydowaliśmy się na funkcję swap, która zamienia miejscami cyfry o pozycjach x i y w bloku 3x3 o numerze blockNo.
	\begin{lstlisting}[language=C++]
	bool swap(unsigned int blockNo, unsigned int x, unsigned int y)
	{
	(...)
	}
	\end{lstlisting}
	
	Dodatkowo, powinniśmy także zdefiniować strukturę, która będzie reprezentować możliwe do wykonania ruchy.
	
	\begin{lstlisting}[language=C++]
	class PossibleMove
	{
	unsigned int blockNo;
	unsigned int x;
	unsigned int y;
	};
	\end{lstlisting}
	
	Po wygenerowaniu planszy uruchamiamy algorytm przeszukiwania z tabu, którego uproszczona wersja jest przedstawiona w poniższym kodzie. Jeszcze niżej jest natomiast opisane co dzieje się w kolejnych krokach algorytmu.
	\begin{lstlisting}[language=C++]
	Sudoku tabuSearchSudokuSolver(Sudoku inputSudoku)
	{
	Sudoku currentSolution = inputSudoku.fillHolesRandomly();	// 1.
	Sudoku bestSolution = currentSolution;	// 2.
	queue<PossibleMove> tabuList;	// 2.
	while (!stopCondition())		// 3.
	{
	Sudoku bestCandidateSolution;
	PossibleMove bestCandidateMove;
	vector<pair<PossibleMove, Sudoku>> currentSolutionNeighbourhood = 							
	currentSolution.getNeighbourhood();	// 4.
	for (auto candidate : currentSolutionNeighbourhood)
	{
	PossibleMove candidateMove = candidate.getKey();
	Sudoku candidateSolution = candidate.getValue();
	if ( !tabuListContainsMove(tabuList, candidateMove) 
	&& (candidateSolution.fitness() > bestCandidateSolution.fitness()) )
	{
	bestCandidateSolution = candidateSolution;	// 5.
	bestCandidateMove = candidateMove;	// 5.
	}
	}
	
	if (bestCandidateSolution.fitness() > bestSolution.fitness())
	{
	bestSolution = bestCandidateSolution;	// 6.
	}
	currentSolution = bestCandidateSolution;	// 7.
	tabuList.push(bestCandidateMove);	// 7.
	if (tabuList.size() > MAX_TABU_LIST_SIZE)
	{
	tabuList.pop();		// 8.
	}
	}
	return bestSolution;	// 9.
	}
	\end{lstlisting}
	\begin{enumerate}
		\item Wypełniamy puste pola planszy wejściowej losowymi liczbami, tak aby w każdym bloku 3x3 planszy znajdowawły się różne cyfry od 1 do 9. Jest to nasze obecne najlepsze rozwiązanie (oczywiście najprawdopodobniej nieprawidłowe).
		\item Tworzymy pustą listę tabu, reprezentowaną jako kolejka FIFO.
		\item Sprawdzamy warunek stopu. W naszym algorytmie funkcja stopCondition() zwróci wartość true wtedy i tylko wtedy, gdy liczba kolizji (czyli liczba powtarzających się cyfr w kolumnach i wierszach) będzie równa 0.
		\item Zczytujemy do wektora sąsiędztwo aktualnego rozwiązania, reprezentowane jako lista par - możliwy ruch i utworzone przez ten ruch sudoku.
		\item Dla każdego elementu z sąsiedztwa, sprawdzamy czy ruch zawiera znajduje się na liście tabu i czy jego fitness jest większy niż fitness najlepszego dotychczasowego kandydata. Funkcja fitness() zwraca tym większą wartość im mniej kolizji występuje. Jeśli tak, to nadpisujemy najlepszego dotychczasowego kandydata znalezionym rozwiązaniem.
		\item Jeśli fitness naszego najlepszego kandydata w danej iteracji jest większy niż fitness najlepszego znalezionego rozwiązania od początku działania algorytmu, to nadpisujemy to najlepsze rozwiązanie.
		\item Nadpisujemy obecne rozwiązanie najlepszym kandydatem i wrzucamy wykonany ruch na listę tabu.
		\item Jeśli lista tabu jest już pełna to usuwamy ostatni jej element;
		\item Zwracamy najlepsze dotychczasowe rozwiązanie. Kończymy algorytm.
	\end{enumerate}
\end{document}
